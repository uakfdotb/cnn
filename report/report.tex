\documentclass[10pt,twocolumn]{article}
\title{Convolutional Neural Networks for Classifying Handwritten Digits}
\date{December 5, 2014}
\author{
	Favyen Bastani \\
	fbastani@perennate.com
	\and
	Zhengli Wang \\
	wzl@mit.edu
}

\usepackage[margin=1in]{geometry}
\usepackage{graphicx}
\usepackage{amsmath}
\usepackage{subcaption}
\setlength{\columnsep}{3em}

\begin{document}

\maketitle

\section{Introduction} \label{sec:introduction}

Image recognition algorithms have become pervasive: on Google Maps Street View, image recognition protects privacy by automatically blurring faces and license plate numbers [1]; robots use object recognition to autonomously move or assemble structures; meanwhile, individuals routinely employ optical character recognition for scanning documents and facial recognition for labeling photographs. Over the last decade, image recognition approaches have developed dramatically. Initially, feature extraction techniques such as HOG and FAST were typically combined with support vector machine (SVM) classifiers, with research often focusing on selection of SVM kernels and defining features for specific image types. Recently, though, neural networks have received renewed attention and achieved high classification performance.

% 1 http://www.australianscience.com.au/research/google/35481.pdf

Cells in the mammalian visual cortex are able to identify local patterns in the visual field [2]. Convolutional neural networks (CNNs) aim to do the same through \emph{convolutional and pooling layers}. While traditional neural networks are composed of densely connected layers, where each neuron in the previous layer connects to each neuron in the next layer, convolutional layers restrict filters to be applied on local sub-regions of an input image. By training these layers through back-propogation, these filters eventually detect specific features in the image that help with classification. CNNs have achieved high image recognition accuracy on a large number of datasets [ref].

% 2 http://deeplearning.net/tutorial/lenet.html

In this project, we implement a fast CPU-based CNN toolkit in Python using the \texttt{numpy} scientific computing library, and apply it to the MNIST handwritten digits database [3]. We evaluate the performance of various neural network parameters (including layer configuration, output layer activation function, and training algorithm) in terms of classification accuracy and training speed, and compare CNN to feature extraction approaches. We find that while CNNs require significantly more training time, they outperform other algorithms with minimal tuning of the network.

% 3 http://yann.lecun.com/exdb/mnist/

In Sections \ref{sec:neural} and \ref{sec:cnn}, we detail convolution and pooling functions, the convolutional neural network structure and training algorithm, and our implementation. In Section \ref{sec:evalcnn}, we experiment with various neural network parameters, and in Section \ref{sec:evalfeature}, we compare with SVM-based algorithms. Finally, we conclude in Section \ref{sec:conclude}.

\section{Neural networks} \label{sec:neural}

An artificial neural network (ANN) consists of neuron units that, when composed in a layered network, compute some output function from an input vector $y = h(x)$. Each neuron in the network has a specific set of input values (which may be elements of the input vector of the network, or outputs from other neurons) and a single output. The output is generally computed as $n_{w, b}(z) = f(w \cdot z + b)$ for some \emph{activation function} $f$, weight vector $w$, and bias parameter $b$. The sigmoid function is often used as the activation function, i.e. $f(l) = \frac{1}{1+e^{-l}}$, because it mimics the hard threshold activation pattern of a biological neuron.

Neurons are connected to form a neural network. Typically, the network is composed of layers of neurons, where neurons in the first layer accept inputs from the input vector $x$, and neurons in each following layer take inputs from the outputs of the neurons in the preceding layer. Then, the network output vector $y$ consists of the outputs from neurons in the last layer of the network. In fully connected neural networks, a neuron takes inputs from every neuron in the previous layer.

Forward propogation refers to the process of propogating some input vector $x$ through each layer of the neural network to compute $y = h(x)$. In a fully connected neural network, this can be done using efficient matrix operations (which can take advantage of fast matrix libraries such as \texttt{numpy}). Let $Z_l$ be the output vector of the $l$th neural network layer (where $Z_0 = x$) and $n_{li}$ be the $i$th neuron in layer $l$; then, $Z_{l+1} = f(W^{\text{T}} Z_l + B)$ where $W_{ij}$ is the weight from $n_{li}$ to $n_{l+1,j}$ and $B_j$ is the bias parameter for neuron $n_{l+1,j}$.

Neural networks can be trained using back propogation. At a high level, we compute a cost function on the outputs from the neural network given some input and desired output, and propogate the error (in the form of partial derivatives of the cost) starting at the last layer to the first layer. These error terms are then used to compute the derivative with respect to specific weight and bias parameters, which can then be updated via gradient descent.

\subsection{ANN for multi-label classification}

Artificial neural networks are often used for multi-label classification. However, the sigmoid activation function is not well suited for this purpose, where we want to assign exactly one label $c$ for each input vector $x$. Instead, we use \emph{softmax activation function}, defined as

$$f(L, i) = \frac{e^{L_i}}{\sum_j e^{L_j}}$$

Here, $L$ is a vector containing inputs to the activation function for each neuron in the softmax layer. For back propogation, the cost function given a desired output vector $o$ is the cross-entropy error function,

$$E = -\sum_i o_i \text{log}(y_i)$$

% http://www.willamette.edu/~gorr/classes/cs449/classify.html

\subsection{Stochastic gradient descent}

In normal gradient descent neural network training, we iteratively update parameters (weights and bias terms) after averaging derivatives from back propogation across all training samples. However, when the training set is large, each iteration of gradient descent may take a significant amount of time to complete. Stochastic gradient descent instead randomly selects mini-batches of a predetermined size from the training samples on each iteration and updates paramaters by averaging derivatives across only the selected samples (the batch size is generally selected to yield optimal performance with matrix libraries, although in our implementation we perform back propogation separately for each sample). This increases solution sparsity as well as convergence speed.

We implement one variation of stochastic gradient descent that further reduces training time in many cases. Rather than updating parameters directly from the computed gradient $\nabla$, we apply a momentum that retains gradients from previous iterations. Let $v_0 = 0$ be the initial velocity. For a training rate $\alpha$ and momentum $m$, we update parameters $\theta_i$ on the $i$th iteration as

% http://ufldl.stanford.edu/tutorial/supervised/OptimizationStochasticGradientDescent/

$$v_i = mv_{i-1} + \alpha \nabla_i$$
$$\theta_i = \theta_{i-1} - v$$

Note that the training rate $\alpha$ is generally smaller in stochastic gradient descent than in conventional gradient descent because the random mini-batch selection introduces greater variance on the training process.

\section{Convolutional neural networks} \label{sec:cnn}

Images, along with many other data types such as audio signals, exhibit strong local patterns that need to be identified for classification. In some cases, fully connected neural networks may ignore these patterns, perhaps overfitting instead; in other cases, training fully connected networks to identify local patterns and achieve high classification accuracy takes a significant number of training iterations. Convolutional neural networks take an alternate approach, where the localized pattern properties are directly encoded into the structure of the neural network. In this section, we discuss the functionality of convolution and pooling layers, and strategies for incorporating them into neural networks.

\subsection{Convolution}

Images can often be hierarchically broken down into features and sub-features that support image recognition. For example, a flower may contain a stem and a petal, each of which consist of edges at various positions and angles. Additionally, these features may exist at various locations in the image: if an image is shifted in one direction, an image recognition algorithm should still be able to classify it in the same way.

The idea behind convolutional layers for image recognition is that a feature can be represented as an $n$ by $n$ \emph{filter}, and then convolved two-dimensionally with the $m$ by $m$ input (which in this case would be an image). If regions of the input match up with the filter, then elements of the convolution output corresponding to those regions will generally take higher values. The output from the convolution is added with a bias term and passed through an activation function. Notably, each component of the convolution can be modeled as a neuron: each neuron takes inputs from a sub-region, multiplies the input by a weight matrix (the filter), and then applies the activation function; besides the sparsely connected input, the only other difference from conventional neural networks is that we require the weight matrix to be the same across multiple neurons (since convolution applies a single filter repeatedly across the image). This enables the detection of local patterns that we desire, since the filter is restricted to only consider sub-regions of the input, while still incorporating the weights and bias components that define neural networks.

In each convolutional layer, we have $k$ filters, each an $n$ by $n$ matrix. Then, for an $m$ by $m$ input, we get a $k$ by $(m - n + 1)$ by $(m - n + 1)$ output from the two-dimensional convolution.

\subsection{Pooling}

Pooling layers apply an aggregate function (typically either max or mean) on non-overlapping regions of the output from convolutional layers (this is done separately for each filter). These regions are often equal in size; so if the output from the convolutional layer is 30 by 30, the pooling layer may pool across each of the four 15 by 15 regions, yielding four output values.

Pooling accomplishes two purposes. First, we reduce the amount of data that later layers need to consider, speeding up training and reducing overfitting. Second, we get a degree of translation invariance. If the input is shifted slightly, the result after pooling will still be similar. This is desirable since generally image recognition classes should remain the same under basic transforms.

\subsection{Network}

Convolution and pooling can be used in isolation, without incorporation in a convolutional neural network. In this case, an autoencoder is typically used to train the filters, where the training goal is to have the output layer match the input layer despite a smaller number of neurons in intermediate layers. The trained filters can then be used to extract features from both training and testing data, and then fed into any classifier. This also supports unsupervised learning of features.

Convolutional neural networks present a more interesting case, where the convolution and pooling operations are combined with classification into a single structure that can be trained in one process. Typical convolutional neural networks consist of one to three convolutional layers followed by fully connected hidden layers; back propogation is carried out across the entire network. When there are several convolutional layers, the first convolutional layer often is used by the network as an edge detector (this determination happens from training alone, without any manual specification other than the layer sizes), while later convolutional layers in the network detect more and more complex features; for simple datasets, a single convolutional layer can be applied to directly identify features. The hidden layers use final pooling outputs to eventually classify the image.

\section{CNN evaluation} \label{sec:evalcnn}

We implement a CPU-based convolutional neural network library in Python using the \texttt{numpy} package, which provides fast scientific computing operations. We evaluate its performance on the MNIST dataset of handwritten digits in terms of overall training algorithm, network structure, number of filters, size of filters, number of pooling regions, and output layer activation function.

\subsection{MNIST dataset}

The Mixed National Institute of Standards and Technology (MNIST) dataset contains sixty thousand training images and ten thousand testing images. Its name comes from the NIST dataset that it is an adaptation of. In the NIST dataset, training and testing images were taken from two different sources, while in MNIST, the sources are merged.

Each MNIST image contains a single handwritten digit (from zero to nine). The images are all 28 pixels by 28 pixels and grayscale.

\section{Comparison with SVM-based recognition} \label{sec:evalfeature}

\section{Conclusion} \label{sec:conclude}

% future work: dropout

\end{document}
